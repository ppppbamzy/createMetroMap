%!TEX program = xelatex

\documentclass[border=1cm]{standalone}

\usepackage{xeCJK}

\usepackage{tikz}
\usetikzlibrary{shapes,positioning,fit}

\setmainfont{Avenir-Black}
\setCJKmainfont{苹方-简 中粗体}

\begin{document}
            \begin{tikzpicture}[node distance=48pt, every node/.append style={rounded corners, inner sep=2}]
            \coordinate (plc1) {};
            \coordinate [below = of plc1] (plc2) {};
            \coordinate [above left = of plc1] (plc3) {};
            \coordinate [below left = of plc2] (plc4) {};
            \coordinate [below right = of plc2] (plc5) {};
            \coordinate [above right = of plc1] (plc6) {};
            \coordinate [right = of plc6] (plc7) {};
            \coordinate [right = of plc5] (plc8) {};
            \coordinate [above = of plc7] (plc9) {};    
            \coordinate [below = of plc8] (plc10) {};
            
            
            \draw[blue, very thick] (plc9) -- (plc7);
            \draw[blue, very thick] (plc10) -- (plc8);
            \draw[blue, very thick] (plc6) -- (plc7);
            \draw[blue, very thick] (plc5) -- (plc8);
            \draw[blue, very thick] (plc1) to [bend right = 45] (plc6);
            \draw[blue, very thick] (plc2) to [bend left = 45] (plc5);
            \draw[blue, very thick] (plc1) to [bend left = 45] (plc3);
            \draw[blue, very thick] (plc2) to [bend right = 45] (plc4);
            \draw[blue, very thick] (plc3) to [bend right = 90] (plc4);
            
            \node [draw, very thick, fill=white, align = center] (plc1_num) {G\\04};
            \node [below = of plc1, anchor = center, draw, very thick, fill=white, align = center] (plc2_num) {G\\07};
            \node [above left = of plc1, anchor = center, draw, very thick, fill=white, align = center] (plc3_num) {G\\05};
            \node [below left = of plc2, anchor = center, draw, very thick, fill=white, align = center] (plc4_num) {G\\06};
            \node [below right = of plc2, anchor = center, draw, very thick, fill=white, align = center] (plc5_num) {G\\08};
            \node [above right = of plc1, anchor = center, draw, very thick, fill=white, align = center] (plc6_num) {G\\03};
            \node [right = of plc6, anchor = center, draw, very thick, fill=white, align = center] (plc7_num) {G\\02};
            \node [right = of plc5, anchor = center, draw, very thick, fill=white, align = center] (plc8_num) {G\\09};
            \node [above = of plc7, anchor = center, draw, very thick, fill=white, align = center] (plc9_num) {G\\01};   
            \node [below = of plc8, anchor = center, draw, very thick, fill=white, align = center] (plc10_num) {G\\10};  
            
            
            \node [right = 0 of plc7_num] (plc7_word) {第二}; 
            \node [right = 0 of plc9_num] (plc9_word) {第一}; 
            \node [right = 0 of plc10_num] (plc10_word) {第十}; 
            \node [right = 0 of plc8_num] (plc8_word) {第九}; 
            \node [right = 0 of plc1_num, yshift=-4] (plc1_word) {第四}; 
            \node [right = 0 of plc2_num, yshift=4] (plc2_word) {第七}; 
            \node [left = 0 of plc3_num, yshift=4] (plc3_word) {第五}; 
            \node [left = 0 of plc4_num, yshift=-4] (plc4_word) {第六}; 
            \node [below = 0 of plc5_num, align = center, execute at begin node=\setlength{\baselineskip}{0ex}] (plc5_word) {第\\八};
            \node [above = 0 of plc6_num, align = center, execute at begin node=\setlength{\baselineskip}{0ex}] (plc6_word) {第\\三}; 

            
            \end{tikzpicture}
\end{document}